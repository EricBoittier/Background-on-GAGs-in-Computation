%%%%%%%%%%%%%%%%%%%%%%%%%%%%%%%%%%%%%%%%%%%%%%%%%%%%%%%%%%%%%%%%%%%%%
%% This is a (brief) model paper using the achemso class
%% The document class accepts keyval options, which should include
%% the target journal and optionally the manuscript type.
%%%%%%%%%%%%%%%%%%%%%%%%%%%%%%%%%%%%%%%%%%%%%%%%%%%%%%%%%%%%%%%%%%%%%
\documentclass[journal=jacsat,manuscript=article]{achemso}

%%%%%%%%%%%%%%%%%%%%%%%%%%%%%%%%%%%%%%%%%%%%%%%%%%%%%%%%%%%%%%%%%%%%%
%% Place any additional packages needed here.  Only include packages
%% which are essential, to avoid problems later. Do NOT use any
%% packages which require e-TeX (for example etoolbox): the e-TeX
%% extensions are not currently available on the ACS conversion
%% servers.
%%%%%%%%%%%%%%%%%%%%%%%%%%%%%%%%%%%%%%%%%%%%%%%%%%%%%%%%%%%%%%%%%%%%%
\usepackage[version=3]{mhchem} % Formula subscripts using \ce{}
\usepackage{natbib}

%%%%%%%%%%%%%%%%%%%%%%%%%%%%%%%%%%%%%%%%%%%%%%%%%%%%%%%%%%%%%%%%%%%%%
%% If issues arise when submitting your manuscript, you may want to
%% un-comment the next line.  This provides information on the
%% version of every file you have used.
%%%%%%%%%%%%%%%%%%%%%%%%%%%%%%%%%%%%%%%%%%%%%%%%%%%%%%%%%%%%%%%%%%%%%
%%\listfiles

%%%%%%%%%%%%%%%%%%%%%%%%%%%%%%%%%%%%%%%%%%%%%%%%%%%%%%%%%%%%%%%%%%%%%
%% Place any additional macros here.  Please use \newcommand* where
%% possible, and avoid layout-changing macros (which are not used
%% when typesetting).
%%%%%%%%%%%%%%%%%%%%%%%%%%%%%%%%%%%%%%%%%%%%%%%%%%%%%%%%%%%%%%%%%%%%%
\newcommand*\mycommand[1]{\texttt{\emph{#1}}}

%%%%%%%%%%%%%%%%%%%%%%%%%%%%%%%%%%%%%%%%%%%%%%%%%%%%%%%%%%%%%%%%%%%%%
%% Meta-data block
%% ---------------
%% Each author should be given as a separate \author command.
%%
%% Corresponding authors should have an e-mail given after the author
%% name as an \email command. Phone and fax numbers can be given
%% using \phone and \fax, respectively; this information is optional.
%%
%% The affiliation of authors is given after the authors; each
%% \affiliation command applies to all preceding authors not already
%% assigned an affiliation.
%%
%% The affiliation takes an option argument for the short name.  This
%% will typically be something like "University of Somewhere".
%%
%% The \altaffiliation macro should be used for new address, etc.
%% On the other hand, \alsoaffiliation is used on a per author basis
%% when authors are associated with multiple institutions.
%%%%%%%%%%%%%%%%%%%%%%%%%%%%%%%%%%%%%%%%%%%%%%%%%%%%%%%%%%%%%%%%%%%%%
\author{Eric D. Boittier}
\email{eric.boittier@uqconnect.edu.au}
\affiliation[UQ]
{The University of Queendland, St Lucia, Queensland, Australia}


%%%%%%%%%%%%%%%%%%%%%%%%%%%%%%%%%%%%%%%%%%%%%%%%%%%%%%%%%%%%%%%%%%%%%
%% The document title should be given as usual. Some journals require
%% a running title from the author: this should be supplied as an
%% optional argument to \title.
%%%%%%%%%%%%%%%%%%%%%%%%%%%%%%%%%%%%%%%%%%%%%%%%%%%%%%%%%%%%%%%%%%%%%
\title[Honours]
  {Computational studies of glycosaminoglycans}

%%%%%%%%%%%%%%%%%%%%%%%%%%%%%%%%%%%%%%%%%%%%%%%%%%%%%%%%%%%%%%%%%%%%%
%% Some journals require a list of abbreviations or keywords to be
%% supplied. These should be set up here, and will be printed after
%% the title and author information, if needed.
%%%%%%%%%%%%%%%%%%%%%%%%%%%%%%%%%%%%%%%%%%%%%%%%%%%%%%%%%%%%%%%%%%%%%
\abbreviations{IR,NMR,UV}
\keywords{American Chemical Society, \LaTeX}

\begin{document}
%%%%%%%%%%%%%%%%%%%%%%%%%%%%%%%%%%%%%%%%%%%%%%%%%%%%%%%%%%%%%%%%%%%%%
%% The manuscript does not need to include \maketitle, which is
%% executed automatically.  The document should begin with an
%% abstract, if appropriate.  If one is given and should not be, the
%% contents will be gobbled.
%%%%%%%%%%%%%%%%%%%%%%%%%%%%%%%%%%%%%%%%%%%%%%%%%%%%%%%%%%%%%%%%%%%%%

\begin{abstract}
Glycosaminoglycans
\end{abstract}
\renewcommand{\contentsname}{Table of Contents}



% TABLE OF CONTENTS
% \newpage
% \tableofcontents
% \newpage





%%%%%%%%%%%%%%%%%%%%%%%%%%%%%%%%%%%%%%%%%%%%%%%%%%%%%%%%%%%%%%%%%%%%%
%% Start the main part of the manuscript here.
%%%%%%%%%%%%%%%%%%%%%%%%%%%%%%%%%%%%%%%%%%%%%%%%%%%%%%%%%%%%%%%%%%%%%
\pagebreak
\section{Introduction}
\addcontentsline{toc}{section}{Introduction}
Protein-ligand interactions have garnered significant attention in the past few decades\cite{Hricovíni2007}. 

Glycosaminoglycans are complex carbohydrate molecules that interact with a wide variety of proteins.\cite{Gandhi2008, Hricovíni2007}

GAGs are challenging from a molecular modeling perspective because of their high negative charge density, their conformational flexibility, and the absence of well-defined binding pockets or high surface complementarities on their target protein. The accurate computational prediction of the free energy of the interaction of sulfated GAG-protein complexes is still in its infancy, particularly because of the poorly defined contribution of water (solvation/desolvation), the large electrostatic interactions involved, and limitations in the force fields and scoring functions used to represent GAG structure, dynamics, and interactions.\cite{Gandi2009}

The Monte Carlo multiple minima (MCMM) method (Keser ̈ u and Kolossv ́ ary 1999) has been used to sample the many degrees of conformational freedom present in large GAG molecules as part of an investigation into the possible binding modes of cy-clitols (GAG-like sulfated molecules) on the fibroblast growth factors 1 and 2 (FGF-1 and FGF-2) (Cochran et al. 2005). \cite{Gandi2009}

Accurate prediction of free energies of the binding of GAGs will require further development and parameterization of the force fields used, particularly if an appropriate description of the likely polarization effects in these systems (due to their high charge density) is to be achieved. All GLYCAM04 charges are developed from a thermally derived ensemble of conformations from long simulations performed in the presence of explicit solvent to represent the average behavior of the molecule in solution (Basma et al. 2001). \cite{Gandi2009}

The  molecular  mechanics  Poisson–Boltzmann  surface  area (MM-PBSA) method was developed to estimate the Gibbs free energy of interactions between proteins and ligands. \cite{Gandi2009}

Using the MM-PBSA method, free  energy  calculations  reveal  that  the  binding  of  heparin  to  protein  surfaces  is  dominated  by  strong  electrostatic  interactions  for  longer  fragments,  with  equally  important contributions from van der Waals interactions and vibrational  entropy  changes,  against  a  large  unfavorable desolvation penalty due to the high charge density of these molecules. \cite{Gandi2009}

MM-GBSA (molecular mechanics generalized  Born  surface  area)  calculations  (where  electro-static  calculations  are  performed  using  the  generalized  Born approach) (Tsui and Case 2001) have been used to investigate how electrostatic interactions dictate the high affinity of anionic carbohydrates  such  as  Gal-β -(1,4)-GlcNAc  for  galectin-1.  \cite{Gandi2009}

The GLYCAM force field has been found to represent glycosidic linkages and conformer ensembles in good agreement with those estimated by NMR determinations for heparin fragments (Angulo et al. 2003; Zhang et al. 2008). We used a similar protocol to perform unrestrained MD simulations in explicit water for heparin fragments bound to the proteins. The Parm94 (Cornell et al. 1995) force field in AMBER 9.0 (Case et al. 2005) was used with the GLYCAM04 extension for carbohydrates (Woods et al. 1995) in all MD simulations. Existing nonbonded parameters for sulfates and sulfamates were used (Huige and Altona 1995). Force constants for bond lengths and angles as well as torsional parameters that were not available for sulfates were approximated by taking those for phosphates available in the GLYCAM04 force field. Such approximation has been successfully applied to reproduce geometries of heparin oligosaccharides in gas phase simulations (Jin et al. 2005). \cite{Gandi2009} 

Partial atomic charges for the heparin di-, penta-, and octasaccharides were obtained using the restricted electrostatic potential (RESP) method (Bayly et al. 1993; Cornell et al. 1993) with the leap and sander modules in Amber 9.0. For this purpose, all molecules were initially subjected to a full geometry optimization with a 6–31G* basis set using Gaussian 98 (Frisch et al. 1998). A SCF convergence criterion of 10−8 kcal/mol and a “tight” optimization threshold were used. The resulting minimum energy conformation of each saccharide was then subjected to a single point energy calculation with a 6–31G* basis set and the POP = CHelpG charge option.



\subsection{Outline}
Text



%%%%%%%%%%%%%%%%%%%%%%%%%%%%%%%%%%%%%%%%%%%%%%%%%%%%%%%%%%%%%%%%%%%%%
%% The appropriate \bibliography command should be placed here.
%% Notice that the class file automatically sets \bibliographystyle
%% and also names the section correctly.
%%%%%%%%%%%%%%%%%%%%%%%%%%%%%%%%%%%%%%%%%%%%%%%%%%%%%%%%%%%%%%%%%%%%%
\pagebreak
\bibliography{backgroundOnGags}


\end{document}
