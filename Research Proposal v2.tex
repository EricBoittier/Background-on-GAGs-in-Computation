%%%%%%%%%%%%%%%%%%%%%%%%%%%%%%%%%%%%%%%%%%%%%%%%%%%%%%%%%%%%%%%%%%%%%
%% This is a (brief) model paper using the achemso class
%% The document class accepts keyval options, which should include
%% the target journal and optionally the manuscript type.
%%%%%%%%%%%%%%%%%%%%%%%%%%%%%%%%%%%%%%%%%%%%%%%%%%%%%%%%%%%%%%%%%%%%%
\documentclass[journal=jctcce,manuscript=article]{achemso}
\setkeys{acs}{articletitle=false}

%%%%%%%%%%%%%%%%%%%%%%%%%%%%%%%%%%%%%%%%%%%%%%%%%%%%%%%%%%%%%%%%%%%%%
%% Place any additional packages needed here.  Only include packages
%% which are essential, to avoid problems later. Do NOT use any
%% packages which require e-TeX (for example etoolbox): the e-TeX
%% extensions are not currently available on the ACS conversion
%% servers.
%%%%%%%%%%%%%%%%%%%%%%%%%%%%%%%%%%%%%%%%%%%%%%%%%%%%%%%%%%%%%%%%%%%%%
\usepackage[version=3]{mhchem} % Formula subscripts using \ce{}
\usepackage{chemnum}
\usepackage{multirow}
\usepackage{textgreek}
\usepackage{textcomp}
\usepackage{setspace}
\usepackage{afterpage}
\usepackage{natbib}
\usepackage[acroymn]{glossaries}
\usepackage{tabularx} % for 'tabularx' environment
\usepackage{array}

\usepackage{acro}

\acsetup{hyperref=true}

\usepackage{enumitem}

\usepackage{multicol}
\usepackage{graphicx}
\usepackage{subcaption}

% glossary
\usepackage{hyperref}


\hypersetup{
    colorlinks,
    citecolor=black,
    filecolor=black,
    linkcolor=black,
    urlcolor=black
}



%%%%%%%%%%%%%%%%%%%%%%%%%%%%%%%%%%%%%%%%%%%%%%%%%%%%%%%%%%%%%%%%%%%%%
%% If issues arise when submitting your manuscript, you may want to
%% un-comment the next line.  This provides information on the
%% version of every file you have used.
%%%%%%%%%%%%%%%%%%%%%%%%%%%%%%%%%%%%%%%%%%%%%%%%%%%%%%%%%%%%%%%%%%%%%
%%\listfiles

%%%%%%%%%%%%%%%%%%%%%%%%%%%%%%%%%%%%%%%%%%%%%%%%%%%%%%%%%%%%%%%%%%%%%
%% Place any additional macros here.  Please use \newcommand* where
%% possible, and avoid layout-changing macros (which are not used
%% when typesetting).
%%%%%%%%%%%%%%%%%%%%%%%%%%%%%%%%%%%%%%%%%%%%%%%%%%%%%%%%%%%%%%%%%%%%%
\newcommand*\mycommand[1]{\texttt{\emph{#1}}}

%%%%%%%%%%%%%%%%%%%%%%%%%%%%%%%%%%%%%%%%%%%%%%%%%%%%%%%%%%%%%%%%%%%%%
%% Meta-data block
%% ---------------
%% Each author should be given as a separate \author command.
%%
%% Corresponding authors should have an e-mail given after the author
%% name as an \email command. Phone and fax numbers can be given
%% using \phone and \fax, respectively; this information is optional.
%%
%% The affiliation of authors is given after the authors; each
%% \affiliation command applies to all preceding authors not already
%% assigned an affiliation.
%%
%% The affiliation takes an option argument for the short name.  This
%% will typically be something like "University of Somewhere".
%%
%% The \altaffiliation macro should be used for new address, etc.
%% On the other hand, \alsoaffiliation is used on a per author basis
%% when authors are associated with multiple institutions.
%%%%%%%%%%%%%%%%%%%%%%%%%%%%%%%%%%%%%%%%%%%%%%%%%%%%%%%%%%%%%%%%%%%%%
\author{Eric D. Boittier}
\email{eric.boittier@uqconnect.edu.au}
\affiliation[UQ]{The University of Queendland, St Lucia, Queensland, Australia}

\author{\linebreak Supervisors: A/Professor Vito Ferro}
\affiliation[UQ]{The University of Queendland, St Lucia, Queensland, Australia}

\author{Dr Neha Gandhi}
\affiliation[QUT]{Queensland University of Technology, Gardens Point, Queensland, Australia}


%%%%%%%%%%%%%%%%%%%%%%%%%%%%%%%%%%%%%%%%%%%%%%%%%%%%%%%%%%%%%%%%%%%%%
%% The document title should be given as usual. Some journals require
%% a running title from the author: this should be supplied as an
%% optional argument to \title.
%%%%%%%%%%%%%%%%%%%%%%%%%%%%%%%%%%%%%%%%%%%%%%%%%%%%%%%%%%%%%%%%%%%%%
\title[Honours]
  {Parameterisation of ``VinaCarb" for improved docking of heparin/heparan sulfate  \linebreak \large Research Proposal}

%%%%%%%%%%%%%%%%%%%%%%%%%%%%%%%%%%%%%%%%%%%%%%%%%%%%%%%%%%%%%%%%%%%%%
%% Some journals require a list of abbreviations or keywords to be
%% supplied. These should be set up here, and will be printed after
%% the title and author information, if needed.
%%%%%%%%%%%%%%%%%%%%%%%%%%%%%%%%%%%%%%%%%%%%%%%%%%%%%%%%%%%%%%%%%%%%%
\abbreviations{IR,NMR,UV}
\keywords{American Chemical Society}

\DeclareAcronym{Ido}{
    short = IdoA ,
    long = {\small L}-iduronic acid
}

\DeclareAcronym{Glc}{
    short = Glc ,
    long = {\small D}-glucose
}

\DeclareAcronym{Gal}{
    short = Gal ,
    long = {\small D}-galactose
}

\DeclareAcronym{DFT}{
    short = DFT ,
    long = density functional theory
}

\DeclareAcronym{GlcN}{
    short = GlcN ,
    long = {\small D}-glucosamine
}

\DeclareAcronym{GalA}{
    short = GalA ,
    long = {\small D}-galacturonic acid
}

\DeclareAcronym{GlcA}{
    short = GlcA ,
    long = {\small D}-glucuronic acid
}

\DeclareAcronym{GalN}{
    short = GalN ,
    long = {\small D}-galactosamine
}

\DeclareAcronym{GlcNAc}{
    short = GlcNAc ,
    long = {\small D}-N-acetylglucosamine
}

\DeclareAcronym{GalNAc}{
    short = GalNAc ,
    long = {\small D}-N-acetylgalactosamine
}




\DeclareAcronym{PG}{
    short = PG ,
    long = proteoglycan
}

\DeclareAcronym{ECM}{
    short = ECM ,
    long = extracellular matrix
}

\DeclareAcronym{GAG}{
    short = GAG ,
    long = glycosaminoglycan
}

\DeclareAcronym{Hep}{
    short = Hep ,
    long = heparin
}
\DeclareAcronym{HS}{
    short = HS ,
    long = heparan sulfate
}
\DeclareAcronym{HA}{
    short = HA ,
    long = hyaluronic acid
}
\DeclareAcronym{CS}{
    short = CS,
    long = chondroitin sulfate
}
\DeclareAcronym{KS}{
    short = KS,
    long = keteran sulfate
}
\DeclareAcronym{DS}{
    short = DS,
    long = dermatin sulfate
}


\DeclareAcronym{DoS}{
    short = DoS,
    long = degree of sulfation
}

\DeclareAcronym{SNFG}{
    short = SNFG, 
    long = structural nomenclature for glycans
}


\DeclareAcronym{APP}{
    short = APP,
    long = amyloid precursor protein
}
\DeclareAcronym{Al}{
    short = APP,
    long = amyloid precursor protein
}
\DeclareAcronym{QM}{
    short = QM,
    long = quantum mechanics
}
\DeclareAcronym{MD}{
    short = MD,
    long = molecular dynamics
}

\DeclareAcronym{FGF1}{
    short = FGF1,
    long = fibroblast growth factor 1
}
\DeclareAcronym{FGF2}{
    short = FGF2,
    long = fibroblast growth factor 2
}
\DeclareAcronym{FGF}{
    short = FGF,
    long = fibroblast growth factor 
}
\DeclareAcronym{AT3}{
    short = ATIII,
    long = \textalpha-antithrombin III
}

\DeclareAcronym{RMSD}{
    short = RMSD,
    long = root mean square deviation
}
\DeclareAcronym{PRMSD}{
    short = PRMSD,
    long = pose root mean square deviation
}

\DeclareAcronym{RBD}{
    short = RBD,
    long = rigid body docking
}

\DeclareAcronym{SAR}{
    short = SAR,
    long = structure activity relationship
}

\newcommand{\squeezeup}{\vspace{-2.5mm}}


\begin{document}


%%%%%%%%%%%%%%%%%%%%%%%%%%%%%%%%%%%%%%%%%%%%%%%%%%%%%%%%%%%%%%%%%%%%%
%% The manuscript does not need to include \maketitle, which is
%% executed automatically.  The document should begin with an
%% abstract, if appropriate.  If one is given and should not be, the
%% contents will be gobbled.
%%%%%%%%%%%%%%%%%%%%%%%%%%%%%%%%%%%%%%%%%%%%%%%%%%%%%%%%%%%%%%%%%%%%%
{\setstretch{1.5} 
\renewcommand{\contentsname}{Table of Contents}

\renewcommand{\thesubfigure}{\Alph{subfigure}}

\newpage
\tableofcontents
\newpage
\listoffigures
\newpage
\listoftables
\newpage
\begin{multicols}{2}
{\setstretch{1}
\printacronyms[name={Abbreviations}, list-style={table}]
}
\end{multicols}

%%%%%%%%%%%%%%%%%%%%%%%%%%%%%%%%%%%%%%%%%%%%%%%%%%%%%%%%%%%%%%%%%%%%%
%% Start the main part of the manuscript here.
%%%%%%%%%%%%%%%%%%%%%%%%%%%%%%%%%%%%%%%%%%%%%%%%%%%%%%%%%%%%%%%%%%%%%
\pagebreak

\section{Background}
\subsection{Structure and function of heparin/heparin sulfate}

\setlength{\textfloatsep}{0.5cm}
\begin{figure}[!b]
    \begin{subfigure}[b]{0.7\textwidth}
        \includegraphics[width=10cm]{heparin_heparan.pdf}
        \caption{}
        \label{fig:HSHep}
    \end{subfigure}
    \begin{subfigure}[b]{0.28\textwidth}
        \includegraphics[width=4cm]{ido_confs.pdf}
        \caption{}
        \label{fig:IdoConfs}
    \end{subfigure}
    \caption{The structural and conformations diversity of heparin/heparin sulfate polymers. (A)  (B) Ring conformation equilibrium of iduronic acid.}
\end{figure}

On the surface of all eukariotic cells, highly-sulfated complex polysaccharides known as \acp{GAG} act as emissaries for the reception and modulation of a wide range of proteins, affecting normal physiological processes, such as blood coagulation and neuronal development, as well as serious pathophysiological disorders, such as cancer and Alzhiemer's disease. 
\Ac{Hep} and \ac{HS} are members of the \ac{GAG} family and possess some of the highest negative charge density in all of nature. 
In the clinic, \ac{Hep} is a widely used anticoagulant drug to treat patients with thrombic disorders.\cite{Liu2014ChemoenzymaticHeparin.}
Like all \acp{GAG}, these oligosaccharides are comprised of repeating uronic acid residues (\ac{Ido} or its C5 epimer, \ac{GlcA}, which are \textalpha/\textbeta 1\textrightarrow4-linked, respectively) paired with \textalpha 1\textrightarrow4-linked \ac{GlcN} residues (Figure \ref{fig:HSHep}). 
Each \ac{GAG} can exhibit various sulfation patterns and \ac{DoS} (the number of sulfo groups per disaccharide), which is a product of \textgreater26 proteins involved in \ac{GAG} biosynthesis in the golgi apparatus.\cite{SoaresdaCosta2017SulfationDisorders, Varki2009BiologicalGlycans} 
While \ac{Hep} and \ac{HS} are comprised of similar disaccharide building blocks, their \ac{DoS}, saccharide make up and native location in the cell varies.
\ac{Hep} has higher sulfation levels than HS (\ac{DoS} = 2.6 vs 0.6, respectively). 
Upregulation of \ac{Ido} also varies. Around 9 in 10 of the disaccharide units in \ac{Hep} contain \ac{Ido}, while only 2 in 10 \ac{HS} disaccharide units contain \ac{Ido}. While \ac{HS} occurs in many cell types, heparin is isolated exclusively from mast cells.\cite{Liu2014ChemoenzymaticHeparin., Gandhi2008TheProteins}


The structural encoding ability of \acp{GAG} rivals that of DNA, RNA and proteins.\cite{Gama2006SulfationActivity} The amino sugar can be sulphated at C4, C6, the unsubstituted nitrogen or C3 (rare), and the uronic acid residue can be substituted at a variety of positions -- leading to \textgreater1,000,000 possible substitution patterns for a \ac{GAG} octasaccharide.\cite{Gandhi2008TheProteins, SoaresdaCosta2017SulfationDisorders,Gama2006SulfationActivity} GAG sulfation patterns have been likened to the “sulfation code” and multiple \ac{SAR} studies suggest high specificity in relation to function.\cite{Habuchi2004SulfationCode, Gama2006SulfationActivity} Due to this variation, isolating defined sulfation sequences from biological sources is understandably difficult and is often \cite{Gama2006SulfationActivity}. 

Although \ac{GlcA} and \ac{GlcN} remain in the standard $^{4}C_{1}$ conformation, flexible ring conformations of the other heparin components add further structural complexity to the oligosaccharide and are reasoned to have bioloigical relevance during \ac{GAG}--protein interactions.\cite{Sattelle2013DoesHeparanome} 
The exceptional plasticity of \ac{Ido} ring conformations is unique amongst sugars, and is dependent on its position in the chain, ionic conditions and sulfation pattern.\cite{Capila2002Heparin-proteinInteractions., vanBoeckel1987ConformationalAcid}
At the reducing end, \ac{Ido} has been characterized as existing in an equilibrium between $^{4}C_{1}$, $^{1}C_{4}$ and $^{2}S_{O}$, using $^{1}$H NMR spectroscopy (Figure \ref{fig:IdoConfs}).\cite{Ferro1986EvidenceStudies, vanBoeckel1987ConformationalAcid} 
When \ac{Ido} is inside the chain, the equilibrium favors the $^{1}C_{4}$ and $^{2}S_{O}$ conformations exclusively, due to interactions with the 1\textrightarrow4-linked neighbouring sugar.\cite{vanBoeckel1987ConformationalAcid} 
When \ac{Ido} is sulfated at the C2 position, the $^{2}S_{O}$ conformer may become more prominent to minimize unfavorable 1,3 diaxial interactions in the $^{1}C_{4}$ structure.\cite{Hsieh2016UncoveringSulphateb} However, the barrier for interconversion between the two conformers is low and has little effect on the overall conformation of the oligosaccharide, allowing \ac{Ido} to adopt poses suited for specific interactions between basic residues. \cite{Capila2002Heparin-proteinInteractions.} In the case of the \ac{Hep} and \ac{FGF2} complex, two \ac{Ido} residues are poised in $^{1}C_{4}$ and $^{2}S_{O}$ conformations respectively.\cite{Faham1996HeparinFactor} Ring flipping, which is believed to occur on the \textmu s timescale based on \ac{MD} simulations, has been a proposed as mechanism for selective binding/unbinding. \cite{Sattelle2012DependenceIdopyranosides} 

\Ac{GalNAc}

At physiological pH, all carboxylic acid and sulphate groups are deprotonated. 

% PHI and PSI

% Although the overall helical structure is maintained in the
% FGF-bound HSGAG compared with unbound HSGAG, we observe
% distinct changes in the backbone torsion angles of the oligosaccharide
% chain induced upon protein binding. These changes result in local
% deviations in the helical axis that provide optimal ionic and van der
% Waals contact with the protein. A specific conformation and topological
% arrangement of the HSGAG-binding loops of FGF, on the other
% hand, impose structural constraints that induce the local deviations in
% the HSGAG structure, thereby enabling maximum contact between
% HSGAG and the protein.


\pagebreak
\section{Aims}

\section{Methods}


}
\newpage
{\setstretch{0}
\bibliography{mendeley_v2}
}
\newpage
\section{Appendix}
\end{document}

